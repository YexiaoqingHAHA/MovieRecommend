\documentclass{article} % For LaTeX2e
\usepackage{nips13submit_e,times}
\usepackage{hyperref}
\usepackage{url}
%\documentstyle[nips13submit_09,times,art10]{article} % For LaTeX 2.09


\title{Movie Recommendation based on Collaborative Topic Modeling}


\author{
Abhishek Bhowmick \\
Department of Computer Science\\
Carnegie Mellon University\\
Pittsburgh, PA 15213 \\
\texttt{abhowmi1@andrew.cmu.edu} \\
\And
Udbhav Prasad \\
Department of Computer Science\\
Carnegie Mellon University\\
Pittsburgh, PA 15213 \\
\texttt{udbhavp@andrew.cmu.edu} \\
\And
Satwik Kottur \\
Department of Electrical Engineering\\
Carnegie Mellon University\\
Pittsburgh, PA 15213 \\
\texttt{skottur@andrew.cmu.edu} \\
}

% The \author macro works with any number of authors. There are two commands
% used to separate the names and addresses of multiple authors: \And and \AND.
%
% Using \And between authors leaves it to \LaTeX{} to determine where to break
% the lines. Using \AND forces a linebreak at that point. So, if \LaTeX{}
% puts 3 of 4 authors names on the first line, and the last on the second
% line, try using \AND instead of \And before the third author name.

\newcommand{\fix}{\marginpar{FIX}}
\newcommand{\new}{\marginpar{NEW}}

%\nipsfinalcopy % Uncomment for camera-ready version

\begin{document}

\maketitle

\begin{abstract}
Traditional collaborative filtering relies on reviews provided by viewers
in the movie watching community to make recommendations to the user. In this
project, we attempt to combine this approach with probabilistic topic modeling
techniques to make recommendations that consist not only of movies that are
popular in the community, but also those that are similar in content to movies
that the user has enjoyed in the past.  
\end{abstract}

\section{Introduction}

Recommender systems are an important technology for TV/movie streaming
services like Netflix, Hulu, HBO, Amazon etc. Not only video streaming 
services, but also audio/music streaming sites like Spotify, Pandora make heavy
use of recommender systems. Indeed, any content management system that has large
quantities of information (or the ability to extract such information) such as usage patterns, metadata, natural text etc can and should make use of 
recommendation methods to help find items of interest. Collaborative filtering and content based approaches are the two main methods of solving the problem, 
however each method has its pros and cons. In 
the remaining parts of the paper, we limit ourselves to the problem of movie 
recommendation, however most of the discussion/analysis can be generally applied
to other domains.  

\section{Motivation}

\subsection{Collaborative Filtering and its shortcoming}

Traditional collaborative filtering makes use of interactions between users and 
items. They may be broadly classified into two categories - neighbourhood methods and latent factor models. Neighbourhood models explicitly capture relationships between items (or users) and predict a user's liking for a particular item based on ratings of neighbouring items by the same user. The other approach, latent
factor models, directly characterize both users and items by latent factors. We focus on factor based models as they are more accurate than neighbourhood based 
methods. However, all collaborative filtering methods suffer from the 'cold 
start' problem, that is 
they are unable to recommend movies in the absence of rating patterns. In fact, 
in the domain of music, it has been observed \cite{music-long-tail} that the distribution of available rating information for music artists has a very long tail, meaning that most of the music items have little rating data available. We believe the same is true of movies as well. We would like to be able to recommend movies that are in this long tail. 


%Traditional collaborative filtering techniques make use of usage patterns, 
%or more specifically, movie reviews. Movie ratings provided by a user
%are used alongwith similar ratings by other users to build a model that captues the user's preferences. This model is then used to predict movies that the user
%may have an interest in. However, this method only works when sufficient usage
%data is available. New content that is available may not be possible to recommend in absence of sufficient usage data. This is known as the 'cold start' 
%problem.
%
%One strategy might be to randomly recommend newly arrived movies to users and record their responses, thus building up usage data. However, such an approach has
%a few shortcomings. Since an average user likes only a few types of movies, it 
%is more likely than not for the user to give a negative review to a randomly suggested movie. Getting a sufficient number of positive reviews may take a long
%time through this approach, and also user satisfaction decreases due to the 
%bad recommendations made by the system (assuming a random recommendation is more
%likely to be bad than good). 


\subsection{Content Based Recommendation}

Content-based recommendation addresses the 'cold start' problem associated with 
collaborative filtering, where certain items do not have any rating information 
and hence the corresponding item vectors consist of all zeroes (we use 
zeroes to represent missing ratings in the rating matrix). One approach is to 
use topic modeling on movie plot summaries to identify latent themes/topics. We 
can learn topic representations for each item (a vector of topic proportions) 
and add them to the item vectors in the latent-factor model. Such topic 
representations of movie items are also useful outside the domain
of movie recommendation. Interpretability of topics may help in explaining 
recommendations to users, effective content programming and ad targeting based
on user profiles \cite{fLDA}. 
 
%A simple way is to make
%recommendations based on movie metadata such as genre, actors, language etc. 
%However, this approach severely restricts the pool of movies from which new 
%recommendations can be made. It also leads to very predictable results, 
%since recommendations made on the basis of metadata alone resemble the results
%a user would have got through simple keyword searches.

%gA much more interesting approach is to identify similarities among movies 
%gthrough latent themes extracted from information such as plot summaries. Topic 
%gmodeling can be used to describe movies in terms of such latent themes. Such an
%gapproach can allow the recommendation system to generalize to new movies that 
%ghave very little usage data.

\section{Problem Definition}

Briefly, the problem we are trying to solve is predict how highly a user will 
rate certain movies based on all users' rating histories and plot summaries for 
all movies. Making use of these predicted ratings, we come up with movie 
recommendations for a user. The problem can be formalized as follows 
\cite{grouplens} :

We are given a list of users \textit{U} = $\{u_{1}, u_{2} ... u_{m}\}$ and a 
list of items \textit{V} = $\{v_{1}, v_{2} ... v_{n}\}$, where each user $u_{i}$
has a list of items $I_{u_{i}}$ which he/she has given ratings for. For a given
user $u_{a} \in \textit{U}$, we need to solve the following two tasks:

\begin{itemize}

\item {\bf Prediction}: Estimate the predicted rating $P_{a_{j}}$ of an
item $v_{j} \notin V_{u_{a}}$. 

\item {\bf Recommendation}: Return a list of N items $I_{r} \in I \:\&\: I_{r} \cap I_{a} = \phi$, that the user will like most. This is simply a problem of
returning the items with highest predicted rating values.

\end{itemize}

Specifically, we are interested in observing how incorporation of item topic 
representations increases the prediction accuracy of factor models. We would
also like to analyse the latent themes of the items that are captured by the
topic model.

\section{Proposed Method}
\label{gen_inst}

The text must be confined within a rectangle 5.5~inches (33~picas) wide and
9~inches (54~picas) long. The left margin is 1.5~inch (9~picas).
Use 10~point type with a vertical spacing of 11~points. Times New Roman is the
preferred typeface throughout. Paragraphs are separated by 1/2~line space,
with no indentation.

Paper title is 17~point, initial caps/lower case, bold, centered between
2~horizontal rules. Top rule is 4~points thick and bottom rule is 1~point
thick. Allow 1/4~inch space above and below title to rules. All pages should
start at 1~inch (6~picas) from the top of the page.

%The version of the paper submitted for review should have ``Anonymous Author(s)'' as the author of the paper.

For the final version, authors' names are
set in boldface, and each name is centered above the corresponding
address. The lead author's name is to be listed first (left-most), and
the co-authors' names (if different address) are set to follow. If
there is only one co-author, list both author and co-author side by side.

Please pay special attention to the instructions in section \ref{others}
regarding figures, tables, acknowledgments, and references.

\section{Experiments}
\label{headings}

First level headings are lower case (except for first word and proper nouns),
flush left, bold and in point size 12. One line space before the first level
heading and 1/2~line space after the first level heading.

\subsection{Headings: second level}

Second level headings are lower case (except for first word and proper nouns),
flush left, bold and in point size 10. One line space before the second level
heading and 1/2~line space after the second level heading.

\subsubsection{Headings: third level}

Third level headings are lower case (except for first word and proper nouns),
flush left, bold and in point size 10. One line space before the third level
heading and 1/2~line space after the third level heading.

\section{Conclusions}


\section{Citations, figures, tables, references}
\label{others}

These instructions apply to everyone, regardless of the formatter being used.

\subsection{Citations within the text}

Citations within the text should be numbered consecutively. The corresponding
number is to appear enclosed in square brackets, such as [1] or [2]-[5]. The
corresponding references are to be listed in the same order at the end of the
paper, in the \textbf{References} section. (Note: the standard
\textsc{Bib\TeX} style \texttt{unsrt} produces this.) As to the format of the
references themselves, any style is acceptable as long as it is used
consistently.

As submission is double blind, refer to your own published work in the 
third person. That is, use ``In the previous work of Jones et al.\ [4]'',
not ``In our previous work [4]''. If you cite your other papers that
are not widely available (e.g.\ a journal paper under review), use
anonymous author names in the citation, e.g.\ an author of the
form ``A.\ Anonymous''. 


\subsection{Footnotes}

Indicate footnotes with a number\footnote{Sample of the first footnote} in the
text. Place the footnotes at the bottom of the page on which they appear.
Precede the footnote with a horizontal rule of 2~inches
(12~picas).\footnote{Sample of the second footnote}

\subsection{Figures}

All artwork must be neat, clean, and legible. Lines should be dark
enough for purposes of reproduction; art work should not be
hand-drawn. The figure number and caption always appear after the
figure. Place one line space before the figure caption, and one line
space after the figure. The figure caption is lower case (except for
first word and proper nouns); figures are numbered consecutively.

Make sure the figure caption does not get separated from the figure.
Leave sufficient space to avoid splitting the figure and figure caption.

You may use color figures. 
However, it is best for the
figure captions and the paper body to make sense if the paper is printed
either in black/white or in color.
\begin{figure}[h]
\begin{center}
%\framebox[4.0in]{$\;$}
\fbox{\rule[-.5cm]{0cm}{4cm} \rule[-.5cm]{4cm}{0cm}}
\end{center}
\caption{Sample figure caption.}
\end{figure}

\subsection{Tables}

All tables must be centered, neat, clean and legible. Do not use hand-drawn
tables. The table number and title always appear before the table. See
Table~\ref{sample-table}.

Place one line space before the table title, one line space after the table
title, and one line space after the table. The table title must be lower case
(except for first word and proper nouns); tables are numbered consecutively.

\begin{table}[t]
\caption{Sample table title}
\label{sample-table}
\begin{center}
\begin{tabular}{ll}
\multicolumn{1}{c}{\bf PART}  &\multicolumn{1}{c}{\bf DESCRIPTION}
\\ \hline \\
Dendrite         &Input terminal \\
Axon             &Output terminal \\
Soma             &Cell body (contains cell nucleus) \\
\end{tabular}
\end{center}
\end{table}

\section{Final instructions}

Do not change any aspects of the formatting parameters in the style files.
In particular, do not modify the width or length of the rectangle the text
should fit into, and do not change font sizes (except perhaps in the
\textbf{References} section; see below). Please note that pages should be
numbered.

\section{Preparing PostScript or PDF files}

Please prepare PostScript or PDF files with paper size ``US Letter'', and
not, for example, ``A4''. The -t
letter option on dvips will produce US Letter files.

Fonts were the main cause of problems in the past years. Your PDF file must
only contain Type 1 or Embedded TrueType fonts. Here are a few instructions
to achieve this.

\begin{itemize}

\item You can check which fonts a PDF files uses.  In Acrobat Reader,
select the menu Files$>$Document Properties$>$Fonts and select Show All Fonts. You can
also use the program \verb+pdffonts+ which comes with \verb+xpdf+ and is
available out-of-the-box on most Linux machines.

\item The IEEE has recommendations for generating PDF files whose fonts
are also acceptable for NIPS. Please see
\url{http://www.emfield.org/icuwb2010/downloads/IEEE-PDF-SpecV32.pdf}

\item LaTeX users:

\begin{itemize}

\item Consider directly generating PDF files using \verb+pdflatex+
(especially if you are a MiKTeX user). 
PDF figures must be substituted for EPS figures, however.

\item Otherwise, please generate your PostScript and PDF files with the following commands:
\begin{verbatim} 
dvips mypaper.dvi -t letter -Ppdf -G0 -o mypaper.ps
ps2pdf mypaper.ps mypaper.pdf
\end{verbatim}

Check that the PDF files only contains Type 1 fonts. 
%For the final version, please send us both the Postscript file and
%the PDF file. 

\item xfig "patterned" shapes are implemented with 
bitmap fonts.  Use "solid" shapes instead. 
\item The \verb+\bbold+ package almost always uses bitmap
fonts.  You can try the equivalent AMS Fonts with command
\begin{verbatim}
\usepackage[psamsfonts]{amssymb}
\end{verbatim}
 or use the following workaround for reals, natural and complex: 
\begin{verbatim}
\newcommand{\RR}{I\!\!R} %real numbers
\newcommand{\Nat}{I\!\!N} %natural numbers 
\newcommand{\CC}{I\!\!\!\!C} %complex numbers
\end{verbatim}

\item Sometimes the problematic fonts are used in figures
included in LaTeX files. The ghostscript program \verb+eps2eps+ is the simplest
way to clean such figures. For black and white figures, slightly better
results can be achieved with program \verb+potrace+.
\end{itemize}
\item MSWord and Windows users (via PDF file):
\begin{itemize}
\item Install the Microsoft Save as PDF Office 2007 Add-in from
\url{http://www.microsoft.com/downloads/details.aspx?displaylang=en\&familyid=4d951911-3e7e-4ae6-b059-a2e79ed87041}
\item Select ``Save or Publish to PDF'' from the Office or File menu
\end{itemize}
\item MSWord and Mac OS X users (via PDF file):
\begin{itemize}
\item From the print menu, click the PDF drop-down box, and select ``Save
as PDF...''
\end{itemize}
\item MSWord and Windows users (via PS file):
\begin{itemize}
\item To create a new printer
on your computer, install the AdobePS printer driver and the Adobe Distiller PPD file from
\url{http://www.adobe.com/support/downloads/detail.jsp?ftpID=204} {\it Note:} You must reboot your PC after installing the
AdobePS driver for it to take effect.
\item To produce the ps file, select ``Print'' from the MS app, choose
the installed AdobePS printer, click on ``Properties'', click on ``Advanced.''
\item Set ``TrueType Font'' to be ``Download as Softfont''
\item Open the ``PostScript Options'' folder
\item Select ``PostScript Output Option'' to be ``Optimize for Portability''
\item Select ``TrueType Font Download Option'' to be ``Outline''
\item Select ``Send PostScript Error Handler'' to be ``No''
\item Click ``OK'' three times, print your file.
\item Now, use Adobe Acrobat Distiller or ps2pdf to create a PDF file from
the PS file. In Acrobat, check the option ``Embed all fonts'' if
applicable.
\end{itemize}

\end{itemize}
If your file contains Type 3 fonts or non embedded TrueType fonts, we will
ask you to fix it. 

\subsection{Margins in LaTeX}
 
Most of the margin problems come from figures positioned by hand using
\verb+\special+ or other commands. We suggest using the command
\verb+\includegraphics+
from the graphicx package. Always specify the figure width as a multiple of
the line width as in the example below using .eps graphics
\begin{verbatim}
   \usepackage[dvips]{graphicx} ... 
   \includegraphics[width=0.8\linewidth]{myfile.eps} 
\end{verbatim}
or % Apr 2009 addition
\begin{verbatim}
   \usepackage[pdftex]{graphicx} ... 
   \includegraphics[width=0.8\linewidth]{myfile.pdf} 
\end{verbatim}
for .pdf graphics. 
See section 4.4 in the graphics bundle documentation (\url{http://www.ctan.org/tex-archive/macros/latex/required/graphics/grfguide.ps}) 
 
A number of width problems arise when LaTeX cannot properly hyphenate a
line. Please give LaTeX hyphenation hints using the \verb+\-+ command.


\subsubsection*{Acknowledgments}

Use unnumbered third level headings for the acknowledgments. All
acknowledgments go at the end of the paper. Do not include 
acknowledgments in the anonymized submission, only in the 
final paper. 

\subsubsection*{References}

References follow the acknowledgments. Use unnumbered third level heading for
the references. Any choice of citation style is acceptable as long as you are
consistent. It is permissible to reduce the font size to `small' (9-point) 
when listing the references. {\bf Remember that this year you can use
a ninth page as long as it contains \emph{only} cited references.}

\small{
[1] Alexander, J.A. \& Mozer, M.C. (1995) Template-based algorithms
for connectionist rule extraction. In G. Tesauro, D. S. Touretzky
and T.K. Leen (eds.), {\it Advances in Neural Information Processing
Systems 7}, pp. 609-616. Cambridge, MA: MIT Press.

[2] Bower, J.M. \& Beeman, D. (1995) {\it The Book of GENESIS: Exploring
Realistic Neural Models with the GEneral NEural SImulation System.}
New York: TELOS/Springer-Verlag.

[3] Hasselmo, M.E., Schnell, E. \& Barkai, E. (1995) Dynamics of learning
and recall at excitatory recurrent synapses and cholinergic modulation
in rat hippocampal region CA3. {\it Journal of Neuroscience}
{\bf 15}(7):5249-5262.
}

\end{document}
